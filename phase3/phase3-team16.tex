\documentclass{article}
\usepackage{graphicx,verbatim}

\title{ResTracker Phase 3}
\author{Team 16\\Stephanie Fuller (sfuller@wpi.edu)\\Jamie Bliss (jebliss@wpi.edu)\\Sam
LaFleche (shl@wpi.edu)}

\begin{document}

\maketitle

\section{Requirements}

\begin{itemize}
\item Users are broken up into Administrators, Super Administrators, Anonymous Users, Students, and Groups
\item All users have names, email addresses, and passwords.
\item Students also have a graduation year and major(s).
\item Administrators have titles and Super Administrator privileges or not. 
\item Groups have descriptions and what class (class 1, class 2, class 3, etc.).
\item Events have a name, description, ID, and expected size.
\item Events have sponsoring groups, name, description, and room reservation(s).
\item There are rooms. Rooms have buildings, room numbers, and occupancy.
\item There is equipment. Equipment is a type of equipment.
\item Equipment can be in rooms and can be used for events.
\item When equipment is in a room there is a quantity.
\item A reservation shall include who booked it, what administrator
approved it (if any), when it was booked, when it is being reserved,
and the event it is for.
\item The system notifies the group and user which created it with any
changes to an event or reservation.
\item All changes (eg additions, deletions, and mutations) shall be
logged, including the user who made the change, the event that was
changed, and the old and new data.
\item The database shall be usable by multiple users at once.
\item The database shall have a UI capable of showing relevant information.
\item For each reservation the student who made the reservation must
be a member of the group the reservation is for.
\item Events can have any number of comments on them, including none.
\item Comments must be on exactly one event.
\item Each comment is made only by one user.
\item Events are run by one or more clubs. There cannot be an event run by no
club, but more than one club can run the event. This is as is currently done for
events at WPI.  
\item Groups can run any number of events including none. New clubs can have not
run any events yet. 
\item Events may or may not use equipment.
\item Equipment may or may not be used.
\item Students can make reservations, but do not have to do so.
\item Rooms may be reserved, but do not have to be.
\item Rooms cannot be booked for more than one event at the same time. 
\item Events can be in public, non-reservable places.
\item Events can be in more than one room. Large events can have multiple rooms
reserved for it.
\item Reservations can either be approved or not yet approved.
\item Students can be members of as many clubs as they want to be, including 0.
\item Groups can have as many members as they want, including potentially 0 if
it is a newly created club or members have not yet joined for the year.
\item Anonymous users can view events, rooms, equipment, clubs, reservations, comments, and users.
\item Anonymous users can create user accounts.
\item Student users can view events, rooms, equipment, clubs, reservations, comments, and users.
\item Student users can request group membership.
\item Student users can comment on events and comments.
\item Student users can create events for groups they are members for.
\item Student users can edit the events they create.
\item Student users can make reservations for the events they create.
\item Student users can edit the reservations for the events they create.
\item Group users can view events, rooms, equipment, clubs, reservations, comments, and users.
\item Group users can edit events they sponsor.
\item Group users can edit the reservations for the events they sponsor.
\item Group users can comment on events and comments.
\item Group users can add students to a group.
\item Administrator users can view events, rooms, equipment, clubs, reservations, comments, and users.
\item Administrator users can approve reservations.
\item Administrator users can comment on events and comments.
\item Administrator users can update room information.
\item Super Administrator users can perform any action Administrators can.
\item Super Administrator users can view, add, update and delete: rooms, events, equipment, users, reservations and comments.
\item Users shall be able to search for a room.
\item Users shall be able to get a list of what rooms a group has
reserved and when they are reserved for.
\item Administrators shall be able to view recent changes to the database.
\item An administrator must approve a reservation before it is official.
\end{itemize}


%\section{queries}

%listing page and detail page on including relations:
%\begin{itemize}
%\item room -includes equipment
%\item equipment - includes rooms in
%\item clubs -include events sponsored 
%\item events -include comments, equipment used
%\item users -include events they're in charge of
%\item reservation 
%\end{itemize}

%available times
%overlapping reservations - including events
%search for rooms based on date/time
%search for rooms based on equipment
%search for rooms based on occupancy range (or just minimum occupancy)

%statistic pages:
	%most used rooms
	%favourite events
	%most common majors for running events

%updates:
%clean out sessions
%change approval of reservation
%add a comment
%

\section{Queries}
\subsection{Retrival}

The following items can be queried on for either an identifying list of items,
or for a detailed list of the information and relations in the database about each item:
\begin{itemize}
\item room - includes equipment
\item equipment - includes rooms in
\item clubs - include events sponsored 
\item events - include comments, equipment used
\item users - include events they're in charge of
\item reservation 
\end{itemize}

Rooms can be queried based on available times such that only rooms with no approved reservations at those times are returned.
If there are unapproved reservations for that room at that time that is also returned.

Rooms can be queried based on the presence of specific equipment in them.

Rooms can be queried based on occupancy range (or just minimum occupancy).

Reservations can be queried such that all overlapping reservations are returned along with their events.

A user can request statistic pages from the database. These include statistics on:
\begin{itemize}
\item most commonly used rooms
\item favorite events based on rating of comments about the event
\item which student majors run the most events
\end{itemize}

\subsection{Updates}
\begin{itemize}
\item New users can be added to the database.
\item New reservations can be created.
\item Reservations can be edited.
\item Reservations can be approved.
\item New events can be created.
\item Events can be edited.
\item Rooms, and its contained equipment can be edited.
\item Super Administrators can create, edit, or delete whatever they like,
however only common operations will be supported with prewritten queries.
Others will be handled by given Super Administrators (presumably skilled users) database access.
\end{itemize}

\section{SQL queries}
\subsection{Retrevial}
\verbatiminput{phase3-queries.txt}

\subsection{Updates}

\section{Logs}
\subsection{Retrevial}
We run:
\verbatiminput{phase3-queries.sql} 
into psql and the output is:
\verbatiminput{phase3-log.txt}

\section{Application Layer Decisions}
Python is the language being used to implement the front end of the system. Jamie is familiar with this language
and it is known to be easy to implement web interfaces with it. Jamie has provided Sam and Stephanie with code examples
to help gain familiarity with the language.

PostgresSQL is being used for the database because it is free, like MySQL, but has many advanced features, like ORACLE,
including foreign key checking.

The basic interface opens with a list of upcoming events. From there users can log in and once they have done so they are presented
with links to pages that allow them to query and update the database as is appropriate to their user type. Items related to them,
such as events they are running or need to approve, are also presented on log in.


\section{Advanced Features}
There are a few advanced features that if implemented would be a boon to the system. The use or triggers to detect
when a new reservation conflicts with a previous one and mark them would allow users to tell immediately when there
might be a scheduling problem. It is possible the system may even want to index reservations that conflict with others
as these will be looked at frequently, at least as long as they exist, so that the conflict may be analyzed and resolved.
Triggers could also be used to check database consistency when information is added or removed.

\section{}

\section{Division of Labor}
The project team met 3 times for various lengths of time for this project phase. Most work was done collaboratively,
however the majority of the work on each section is as follows:

Sam:
\begin{itemize}
\item Updating requirements to include different users and how they can use the system.
\item Writing retrieval queries in English.
\item Writing update queries in English.
\item Writing on the use of advanced features like triggers and indexing.
\item Write up of technologies used.
\end{itemize}

Stephanie:
\begin{itemize}
\item Writing retrieval and update queries in SQL.
\item Writing initial test data.
\end{itemize}

Jamie:
\begin{itemize}
\item Testing and updating the database, including addition of more data and running queries.
\item Work on preparing the database and front end for future phases.
\end{itemize}

\section{}

\end{document}

