\documentclass{article}
\usepackage{graphicx,booktabs}

\title{ResTracker \\ Database for WPI club room reservations}
\author{Group 16 \\ Stephanie Fuller(sfuller@wpi.edu) \\ Sam LaFleche
(shl@wpi.edu) \\ Jamie Bliss (jebliss@wpi.edu)}

\begin{document}

\maketitle

\section{Description}
As it currently stands the WPI administration does not have a good tool for
managing what rooms have been reserved by student organizations. Room
reservations are often lost, forgotten about, or double booked leading to
frustration on the part of organizations and difficulty running events. Having a
single database where room reservations could be tracked, added and deleted
would mitigate this problem.

%\section{Requirements}
%%Jamie will finish this later. 
%\begin{itemize}
%\item  Students can create events for clubs and reserve rooms for them. 
%\item Users can comment on events. 
%\item The student who created an event or an administrator can edit an event. 
%\item Administrators shall be able to remove room reserverations should it
%prove necessary. 
%\item Any user shall be able to view information about any event, room, or
%reservation.
%\item A reservation shall include who booked it, what administrator approved it,
%when it was booked, and what it is being used for. 
%\item The system notifies the club and user which created it with any changes to
%an event or reservation.
%\item Students shall be able to search for a room.
%\item Students shall be able to get a list of what rooms a club has reserved and when
%they are reserved for.
%\item All changes (eg additions, deletions, and mutations) shall be logged, including the
%user who made the change, the event that was changed, and the old and new data.
%\item Administrators shall be able to view recent changes to the database, and
%students can view changes to events of their groups. 
%\item The database shall be usable by multiple users at once.
%\item The database shall have various levels of users with various levels of
%privileges. Some users have privileges to read, and some users have privileges to
%read/write.
%\item The database shall have a UI capable of showing relevant information.
%\item For each reservation the student who made the reservation must be a member
%of the group the reservation is for. 
%\end{itemize}

\section{Requirements}
\begin{itemize}
\item Users are broken up into Administrators, Students, and Groups
\item Students and administrators can create events.
\item Events have sponsoring groups, name, description, and room reservation(s).
\item A reservation shall include who booked it, what administrator
approved it (if any), when it was booked, when it is being reserved,
and the event it is for.
\item An administrator must approve a reservation before it is official.
\item Users can comment on events. When they comment on events they can rate the
event.
\item Users can reply to comments. When they reply, they can rate the comment. 
\item The student who created the event, a sponsoring group, or an
administrator can edit an event.
\item Administrators shall be able to remove room reserverations
should it prove necessary.
\item Any user shall be able to view information about any event,
room, or reservation.
\item The system notifies the group and user which created it with any
changes to an event or reservation.
\item Users shall be able to search for a room.
\item Users shall be able to get a list of what rooms a group has
reserved and when they are reserved for.
\item All changes (eg additions, deletions, and mutations) shall be
logged, including the user who made the change, the event that was
changed, and the old and new data.
\item Administrators shall be able to view recent changes to the database
\item Students can view changes to events of their groups.
\item Groups can view changes to events they're sponsoring.
\item The database shall be usable by multiple users at once.
\item The database shall have various levels of users with various
levels of privileges. Some users have privileges to read, and some
users have privileges to read/write.
\item The database shall have a UI capable of showing relevant information.
\item For each reservation the student who made the reservation must
be a member of the group the reservation is for.
\end{itemize}

\section{}
\scalebox{.4}{\includegraphics[angle=90]{ERmodel.eps}}

\section{Additional Constraints}
For each reservation the student who made the reservation must be a member of
the group the reservation is for. 

\section{}
%\begin{matrix}
%users, groups, events, reservations, rooms & -- view, change fields, create,
%query \\
%events & -- add/remove reservations, comment \\
%reservations & -- approve \\
%group & -- add/remove user \\
%users & -- login, logout \\
%\begin{matrix}

\begin{tabular}{lll}
\toprule
 objects & actions & types of users\\
\midrule
 users & login & all \\
users & logout & all \\
users & view & all \\
users & change fields & administator, self \\
users & create & administrator, anonomous \\
users & query/search & all \\
groups & add user & admininstrator, group user\\
groups & remove user & administator, group user, self\\ 
groups & view & all\\
groups & change fields & administator, group user\\
groups & create & administator \\
groups & query/search & all \\
reservations & approve & administator \\
reservations & change fields & administator, event creator, group user \\
reservations & view & all \\
reservations & query/search & all \\
events & remove reservation & administrator, event creator, group user \\
events & add reservation & administator, event creator, group user\\
events & view & all \\
events & change fields & administator, event creator, group user \\
events & query/serach & all \\
\bottomrule
\end{tabular}

\section{}
We came up with design in 5 definitive iterations. 
\begin{enumerate}
\item initial idea
\item breaking out user types
\item adding events
\item adding comments
\item groups are users
\end{enumerate}

Most of the ER model was done by collaberative brainstorming. However, we did
have a long discussion as to whether reservation should be an entity or a
relation.  

We didn't consult any domain experts for our application domain. We did however
use the experience of two of the three of having been officers in a club who had
experience booking events with the current method. 

\section{}
We are groing to have two offical meetings as well as unoffical meetings
throughout the week. We are going to meet after class Thursday in the octowedge
to discuss what we will do for the next week include what individuals will do.
We will also meet at 3 PM on Wedensdays in the lounge on the 3rd floor of
Fuller, so we could go to see the TAs if we find it necessary. While there would
not be other official meetings, we will generally see each other around in the
octowedge and discuss the project at those times. This is doable because we know
each other outside of this class. 

This past week we only met once, Wedensday at 3 PM, however this was because of
the schedule of Stephanie, while she had to deal with the death of her
grandfather. 

We will share our partial results by using a distributed version control system.
We will be using git throughout this project. 
%use git 

\section{}
Generally, all of our work was done jointly. All conceptual work was done
jointly. Stephanie wrote up the ER model, and did less in the
functions and descriptions of the requirements. 

Some examples of specific contributions by different people was Jamie coming up
with the idea of comments, Sam figuring out how to integrate comments
in the system, and Stephanie thinking of changing group to a type of user, and
event to an entity. 

%All conseptual work was joint. 
%Some examples are
%Jamie - comments
%Steph - group as user, event as entity
%Sam - inigrate comments into system 

%Wrote up ER model - Stephanie
%Function ideas - Jamie
%Requirements - Sam 

\end{document}

