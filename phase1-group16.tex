\documentclass{article}
\usepackage{amsmath,amssymb,bbm,graphicx}

\title{ResTracker \\ Database for WPI club room reservations}
\author{Group 16 \\ Stephanie Fuller(sfuller@wpi.edu) \\ Sam LaFleche
(shl@wpi.edu) \\ Jamie Bliss (jebliss@wpi.edu)}

\begin{document}

\maketitle

\section{Description}
As it currently stands the WPI administration does not have a good tool for
managing what rooms have been reserved by student organizations. Room
reservations are often lost, forgotten about, or double booked leading to
frustration on the part of organizations and difficulty running events. Having a
single database where room reservations could be tracked, added and deleted
would mitigate this problem.

\section{Requirements}
%Jamie will finish this later. 
\begin{itemize}
\item  Students can create events for clubs and reserve rooms for them. 
\item Users can comment on events. 
\item The student who created an event or an administrator can edit an event. 
\item Administrators shall be able to remove room reserverations should it
prove necessary. 
\item Any user shall be able to view information about any event, room, or
reservation.
\item A reservation shall include who booked it, what administrator approved it,
when it was booked, and what it is being used for. 
\item The system notifies the club and user which created it with any changes to
an event or reservation.
\item Students shall be able to search for a room.
\item Students shall be able to get a list of what rooms a club has reserved and when
they are reserved for.
\item All changes (eg additions, deletions, and mutations) shall be logged, including the
user who made the change, the event that was changed, and the old and new data.
\item Administrators shall be able to view recent changes to the database, and
students can view changes to events of their groups. 
\item The database shall be usable by multiple users at once.
\item The database shall have various levels of users with various levels of
privileges. Some users have privileges to read, and some users have privileges to
read/write.
\item The database shall have a UI capable of showing relevant information.
\item For each reservation the student who made the reservation must be a member
of the group the reservation is for. 
\end{itemize}

\section{}
\includegraphics{ERmodel.eps}

\section{Additional Constraints}
For each reservation the student who made the reservation must be a member of
the group the reservation is for. 

\section{}
%\begin{matrix}
%users, groups, events, reservations, rooms & -- view, change fields, create,
%query \\
%events & -- add/remove reservations, comment \\
%reservations & -- approve \\
%group & -- add/remove user \\
%users & -- login, logout \\
%\begin{matrix}

\begin{tabular}{ccc}
 objects & actions & types of users\\
 users & login & all \\
users & logout & all \\
users & view & all \\
users & change fields & administator, self \\
users & create & administrator, anonomous \\
users & query/search & all \\
groups & add user & admininstrator, group user\\
groups & remove user & administator, group user, self\\ 
groups & view & all\\
groups & change fields & administator, group user\\
groups & create & administator \\
groups & query/search & all \\
reservations & approve & administator \\
reservations & change fields & administator, event creator, group user \\
reservations & view & all \\
reservations & query/search & all \\
events & remove reservation & administrator, event creator, group user \\
events & add reservation & administator, event creator, group user\\
events & view & all \\
events & change fields & administator, event creator, group user \\
events & query/serach & all \\
\end{tabular}

\section{}
We came up with design in 5 definitive iterations. 
\begin{enumerate}
\item initial idea
\item breaking out user types
\item adding events
\item adding comments
\item groups are users
\end{enumerate}

Most of the ER model was done by collaberative brainstorming. However, we did
have a long discussion as to whether reservation should be an entity or a
relation.  

%Didn't consult, but had officers of different clubs in group. 

\section{}
%after class thursday in octowedge and 3rd floor fuller wed at 3.
%also just generally be around octowedge and talk throughout week.

%this past week we met once

%use git, git-web possibly 

\section{}
%All conseptual work was joint. 
%Some examples are
%Jamie - comments
%Steph - group as user, event as entity
%Sam - inigrate comments into system 

%Wrote up ER model - Stephanie
%Function ideas - Jamie
%Requirements - Sam 

\end{document}

