\documentclass{article}
\usepackage{amsmath,amssymb,bbm}

\title{Project Intent\\Database for WPI club room reservation}
\author{Stephanie Fuller (sfuller@wpi.edu)\\ Sam LaFleche (shl@wpi.edu)\\ Jamie
Bliss (jebliss@wpi.edu)}

\begin{document}

\maketitle

\section{Description}
As it currently stands the WPI administration does not have a good tool for
managing what rooms have been reserved by student organizations. Room
reservations are often lost, forgotten about, or double booked leading to
frustration on the part of organizations and difficulty running events. Having a
single database where room reservations could be tracked, added and deleted
would mitigate this problem.

\section{Requirements}
\begin{itemize}
\item 
Students shall be able to reserve a specific room by date and tine.
\item
Administer shall be able to remove room reservations should it prove
necessary.
\item
Students shall be able to view information about booked rooms.
\item
A booked room shall include who booked it.
\item A booked room shall include what administrator approved it.
\item A booked room shall include when it was booked.
\item A booked room shall include what its being used for.
\item
Students shall be able to search for available to search for a room based on day
free, time free, or specific room.
\item
Students shall be able to get a list of what rooms a club has reserved and when
they are reserved for.
\item
All changes (eg additions, deletions, and mutations) shall be logged, including the
user who made the change, the event that was changed, and the old and new
data.
\item
Users shall be able to view recent changes to the database.
\item
The database shall be usable by multiple users at once.
\item
The database shall have various levels of users with various levels of
privileges. Some users have privileges to read, and some users have privileges to
read/write.
\item
The database shall have a UI capable of showing relevant information.
\end{itemize}

\section{ER description}
This database will have at least the following entities:
\begin{itemize}
\item users
\item groups
\item rooms
\end{itemize}
\\
The following relations are also necessary:
\begin{itemize}
\item reservation - relation between two users, a group, and a room. That is a
room is reserved by someone for a group and a user approved this. 
\item member of - many to many relation between users and groups. That is, users
are members of groups.
\end{itemize}
\\
There will also be some sort of table(s) for logging, but we have not been able to
map this to ER modeling. 

%entities
%users
%groups
%rooms
%relations
%reservations - two users, group, room
%"in group" - many to many
%tables of some sort
%logging

%\section{Expected Implementation}
%\begin{itemize}
%\item 
%
%\end{itemize}

\end{document}

