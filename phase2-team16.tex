\documentclass{article}
\usepackage{graphicx,verbatim}

\title{ResTracker Phase 2}
\author{Team 16\\Stephanie Fuller (sfuller@wpi.edu)\\Jamie Bliss (jebliss@wpi.edu)\\Sam
LaFleche (shl@wpi.edu)}

\begin{document}
%\oddsidemargin -0.5in

\maketitle

%Sam is shl on ishtar

\section{}
%no changes that we came up with
	%run by Jansen?
	%run by Stacy? 
Our requirements were created in an iterative process and ran by the TA before
the submission of phase 1. The TA had no further comments on our phase 1
submission. 

\begin{itemize}
\item Users are broken up into Administrators, Students, and Groups
\item All users have names, email addresses, and passwords.
\item Students also have a graduation year and major(s).
\item Administrators have titles and privledges to remove events or not. 
\item Groups have descriptions and what class (class 1, class 2, class 3, etc.).
\item Events have a name, description, ID, and expected size.
\item Students and administrators can create events.
\item Events have sponsoring groups, name, description, and room reservation(s).
\item There are rooms. Rooms have buildings, room numbers, and occupancy.
\item There is equipment. Equipment is a type of equipment.
\item Equipment can be in rooms and can be used for events.
\item When equipment is in a room there is a quantity.
\item A reservation shall include who booked it, what administrator
approved it (if any), when it was booked, when it is being reserved,
and the event it is for.
\item An administrator must approve a reservation before it is official.
\item Users can comment on events. When they comment on events they can rate the
event. Comments also include when they were made. 
\item Users can reply to comments. When they reply, they can rate the comment. 
\item The student who created the event, a sponsoring group, or an
administrator can edit an event.
\item Administrators shall be able to remove room reserverations
should it prove necessary.
\item Any user shall be able to view information about any event,
room, or reservation.
\item The system notifies the group and user which created it with any
changes to an event or reservation.
\item Users shall be able to search for a room.
\item Users shall be able to get a list of what rooms a group has
reserved and when they are reserved for.
\item All changes (eg additions, deletions, and mutations) shall be
logged, including the user who made the change, the event that was
changed, and the old and new data.
\item Administrators shall be able to view recent changes to the database
\item Students can view changes to events of their groups.
\item Groups can view changes to events they're sponsoring.
\item The database shall be usable by multiple users at once.
\item The database shall have various levels of users with various
levels of privileges. Some users have privileges to read, and some
users have privileges to read/write.
\item The database shall have a UI capable of showing relevant information.
\item For each reservation the student who made the reservation must
be a member of the group the reservation is for.
\item Events can have any number of comments on them, including none.
\item Comments must be on exactly one event.
\item Users may make as many comments as they want, including none.
\item Each comment is made only by one user.
\item Events are run by one or more clubs. There cannot be an event run by no
club, but more than one club can run the event. This is as is currently done for
events at WPI.  
\item Groups can run any number of events including none. New clubs can have not
run any events yet. 
\item Events may or may not use equipment.
\item Equipment may or may not be used. 
\item  Students can make reservations, but do not have to do so.
\item Rooms may be reserved, but do not have to be.
\item Rooms cannot be booked for more than one event at the same time. 
\item Events can be in public, non-reservable places.
\item Events can be in more than one room. Large events can have multiple rooms
reserved for it.
\item Admins can approve as many reservations as they want.
\item Reservations can either be approved or not yet approved.
\item Students can be members of as many clubs as they want to be, including 0.
\item Groups can have as many members as they want, including potentially 0 if
it is a newly created club or members have not yet joined for the year.
\end{itemize}



\section{}
Our ER model was created in an iterative process and ran by the TA before
the submission of phase 1. The TA had no further comments on our phase 1
submission. 
\scalebox{.42}{\includegraphics[angle=90]{ERmodel.eps}}
%\includegraphics[angle=90]{ERmodel.eps}

\section{}
\verbatiminput{EnglishDatabaseTables.txt}

\section{}
Having analyzed every non-key attribute in every table, we realized that no
other attribute is implied by it. This means normalization isn't possible. 

Some things we questioned as to whether or not there were functional
dependences. These were the title and privaledges of an admin, and the title and
description of an event. Because titles are something defined by the
administration, and may not directly relate to how the user uses the system, we
felt there wasn't a functional dependency. In the case of event name and
description, weekly events may have the same name and different descriptions
dependent on what will be going on that week.

\section{}
\verbatiminput{restrackerddl.sql}
\verbatiminput{createtables.log}
%Load data in and show output of script.


\section{}
This past week we met Thursday after class to discuss what we were doing, on
Sunday, and on Wedensday. We are contining to meet on Wedensday to work, when
the phase is first assigned to discuss our plan, and once other, though that
meeting is dependent on our schedules. We also distribute our work over git,
though we meet in person.

\section{}
Like our previous phase, our work was mainly joint. All work was done while we
were in meetings. However, certain parts were mostly done by specific people.
Sam wrote up the english description of the tables of the database. Stephanie
wrote up the ddl statements. Jamie handled the loading of data,configuring the
database, and actually testing the database.

\end{document}

